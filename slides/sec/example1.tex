\section{Testing a single web service operation using jsongen}

% TODO: all tables same width

\begin{frame}{Testing a single web service operation using jsongen}
  \framesubtitle{Operation: new user}
  The main objective of this example is to give a general idea of how
  to use jsongen to test a simple web service operation.\\

  The web service operation:\\

  \centering
  \begin{tabular}{| c | l |}
    \hline
    \textbf{Operation} & new user \\ \hline
    \textbf{URI}       & http://localhost:5000/bank/users/ \\ \hline
    \textbf{Method}    & POST     \\ \hline
    \textbf{Body}      & name: string, password: string \\ \hline
    \textbf{Result}    & user: string \\ \hline
    \textbf{Status}    & 201 \\ \hline
  \end{tabular}
  \centering
\end{frame}

\begin{frame}{Starting out our JSON Schema}
  \centering
  \begin{tabular}{| c | l |}
    \hline
    \textbf{Operation} & new user \\ \hline
    \textbf{URI}       & http://localhost:5000/bank/users/ \\ \hline
    \textbf{Method}    & POST     \\ \hline
  \end{tabular}

  \centering
  \inputminted{js}{./code/example1_op_description.jsch}
\end{frame}

\begin{frame}{Automatic body generation}{Creating the new user body}
  \begin{center}
    \begin{tabular}{| c | l |}
      \hline
      \textbf{Body} & name: string, password: string \\ \hline
    \end{tabular}
  \end{center}
  \begin{tabular}{p{6cm} l}
    JSON Schema definition&JSON generated
  \end{tabular}
  \begin{minipage}{0.5\textwidth}
    \centering
    \inputminted{js}{./code/example1_body.jsch}
  \end{minipage}
  \vline
  \pause
  \hspace{10pt}
  \begin{minipage}{0.4\textwidth}
    \centering
    \inputminted{js}{./code/example1_body-gen.jsch}
  \end{minipage}

  \centering

\end{frame}

\begin{frame}{Body generator: self-defined generators}
  \begin{tabular}{| c | l |}
    \hline
    \textbf{Body} & name: string, password: string \\ \hline
  \end{tabular}
  \centering
  \inputminted{js}{./code/example1_body_custom_generators.jsch}
\end{frame}

\begin{frame}{Response validation}{Validating the new user response}
  \begin{center}
    \begin{tabular}{| c | l |}
      \hline
      \textbf{Result}    & user: string \\ \hline
      \textbf{Status}    & 201 \\ \hline
    \end{tabular}
  \end{center}
  \begin{tabular}{p{6cm} l}
    JSON Schema definition & Valid JSON
  \end{tabular}

  \begin{minipage}{0.5\textwidth}
    \centering
    \inputminted{js}{./code/example1_new_user_response.jsch}
  \end{minipage}
  \vline
  \pause
  \hspace{10pt}
  \begin{minipage}{0.4\textwidth}
    \centering
    \inputminted{js}{./code/example1_new_user_val.jsch}
  \end{minipage}

  \centering
\end{frame}

\begin{frame}{JSON Schema files relationships}
  At the end we will have 2 files:
  \begin{itemize}
  \item \texttt{new\_user.jsch} which contains the information used in
    the request generation.
  \item \texttt{new\_user\_response.jsch} which contains the information
    in the response validation.
  \end{itemize}
  The last important JSON Schema identifier is:
  \inputminted{js}{./code/example1_targetSchema.jsch}
\end{frame}

\begin{frame}{}{}
  \begin{center}
    \vspace{15pt}
    {\Huge Demo}
  \end{center}
\end{frame}