\begin{frame}{Our bank web service}
  \begin{figure}
    \centering
    \begin{overprint}[\textwidth]
    \onslide<1> \includestandalone[mode=buildnew, width=\textwidth]{./figures/figure-api-0}
    \onslide<2> \includestandalone[mode=buildnew, width=\textwidth]{./figures/figure-api-1}
    \onslide<3> \includestandalone[mode=buildnew, width=\textwidth]{./figures/figure-api-2}
    \onslide<4> \includestandalone[mode=buildnew, width=\textwidth]{./figures/figure-api-3}
    \onslide<5> \includestandalone[mode=buildnew, width=\textwidth]{./figures/figure-api-4}
    \onslide<6> \includestandalone[mode=buildnew, width=\textwidth]{./figures/figure-api-5}
    \onslide<7->\includestandalone[mode=buildnew, width=\textwidth]{./figures/figure-api-6}
  \end{overprint}
  \end{figure}
\end{frame}

\begin{frame}{What is jsongen}

  \begin{itemize}
  \item Jsongen is a tool for testing web services based on json
    communication.

  \item We can generate automated and random test cases using
    Quickcheck.

  \item What do we need:
    \begin{itemize}
    \item A JSON Schema for the tested API.\@
    \item Optionally, an Erlang module for state checking.
    \end{itemize}

  \item Differences with other testing tools:
    \begin{itemize}
    \item Automated test cases.
    \item
    \end{itemize}

  \end{itemize}

\end{frame}