\section{Context}

\begin{frame}{What is jsongen}

\end{frame}

\begin{frame}{Our bank web service}
  \begin{figure}
    \centering
      \begin{overprint}[\textwidth]
    \onslide<1>\includestandalone[mode=buildnew,  width=\textwidth]{./figures/figure-api-1}
    \onslide<2> \includestandalone[mode=buildnew, width=\textwidth]{./figures/figure-api-2}
    \onslide<3> \includestandalone[mode=buildnew, width=\textwidth]{./figures/figure-api-3}
    \onslide<4> \includestandalone[mode=buildnew, width=\textwidth]{./figures/figure-api-4}
    \onslide<5> \includestandalone[mode=buildnew, width=\textwidth]{./figures/figure-api-5}
    \onslide<6->\includestandalone[mode=buildnew, width=\textwidth]{./figures/figure-api-6}
  \end{overprint}
  \end{figure}
\end{frame}

\begin{frame}{How to use jsongen}
  What jsongen does:
\begin{itemize}
\item Automatic test case generation.
\item Trazable errors.
\item Extensible library to model service state.
\item Property-based testing of web services.
\end{itemize}

What jsongen needs:
  \begin{itemize}
    \item A JSON Schema specification of the API.
    \item No programming knowledge needed for basic usage.
    \item Erlang knowledge for advanced usage.
  \end{itemize}
\end{frame}

\begin{frame}{Tool demonstration testing a custom web
    service\footnote{Diapositiva prescindible mientras se mencione en la siguiente
    diapositiva}}
  \begin{itemize}
  \item An easy to understand bank web service.
  \item Operations to create resources and modify state.
  \item Jsongen's approach by example.
  \end{itemize}
\end{frame}