\section{Context}

\begin{frame}{How we could test Web services}

  \begin{enumerate}
\item Unit tests.
\item Integration tests. \footnote{*No me convence usar esto, porque no encuentro fácil el paralelismo con jsongen. Me puedo meter donde no me llaman}
\item Model checking tests.

  \end{enumerate}

And common approaches are:
\begin{itemize}
\item Tools for unit testing \footnote{*¿Mencionamos Postman, RESTclient?}
\item Libraries tied to project structure and language.
\item Ad-hoc test framework.
\end{itemize}
\end{frame}

\begin{frame}{How to use jsongen}
  What jsongen does:
\begin{itemize}
\item Automatic test case generation.
\item Trazable errors.
\item Extensible library to model service state.
\item Property-based testing of web services.
\end{itemize}

What jsongen needs:
  \begin{itemize}
    \item A JSON Schema specification of the API.
    \item No programming knowledge needed for basic usage.
    \item Erlang knowledge for advanced usage.
  \end{itemize}
\end{frame}

\begin{frame}{Tool demonstration testing a custom web
    service\footnote{Diapositiva prescindible mientras se mencione en la siguiente
    diapositiva}}
  \begin{itemize}
  \item An easy to understand bank web service.
  \item Operations to create resources and modify state.
  \item Jsongen's approach by example.
  \end{itemize}
\end{frame}