\section{Testing a web service protocol using jsongen}

\begin{frame}{Objectives and API description}
  The main objective of this example is to give a general idea of how
  to use the dynamic links descovering habilities of jsongen.

  In this example we will test the protocol of the whole bank API.

  The API operations:

  \centering
  \begin{tabular}{| l | l |}
    \hline
    \textbf{Operation} & \textbf{Resource identifiers} \\ \hline
    new user           & /bank/users/                  \\ \hline
    new account        & /bank/users/\{user\}/accounts/  \\ \hline
    consult account    & /bank/users/\{owner\}/accounts/\{accountid\}/ \\ \hline
    deposit            & /bank/users/\{owner\}/accounts/\{accountid\}/ \\ \hline
    withdraw           & /bank/users/\{owner\}/accounts/\{accountid\}/ \\ \hline
  \end{tabular}
  \centering

\end{frame}

\begin{frame}{Operation: new account}
  \centering
  \begin{tabular}{| c | l |}
    \hline
    \textbf{Operation} & new account \\ \hline
    \textbf{URI}       & http://localhost:5000/bank/users/\{user\}/accounts/ \\ \hline
    \textbf{Method}    & POST     \\ \hline
    \textbf{Body}      & empty    \\ \hline
  \end{tabular}
  \centering
  \inputminted{js}{./code/example2_new_account.jsch}
\end{frame}

\begin{frame}{Operation: new account}
  \centering
  \begin{tabular}{| c | l |}
    \hline
    \textbf{Result}    & accountid: string, balance: integer, owner: string\\ \hline
    \textbf{Status}    & 201 \\ \hline
  \end{tabular}
  \centering
  \inputminted{js}{./code/example2_new_account_response.jsch}
\end{frame}

\begin{frame}{Operation: consult account}
  \centering
  \begin{tabular}{| c | l |}
    \hline
    \textbf{Operation} & consult account \\ \hline
    \textbf{URI}       & http://localhost:5000/bank/users/\{owner\}/accounts/\{accountid\} \\ \hline
    \textbf{Method}    & GET     \\ \hline
  \end{tabular}
  \centering
  \inputminted{js}{./code/example2_consult_account.jsch}
\end{frame}

\begin{frame}{Operation: consult account}
  \centering
  \begin{tabular}{| c | l |}
    \hline
    \textbf{Result}    & accountid: string, balance: integer, owner: string\\ \hline
    \textbf{Status}    & 200 \\ \hline
  \end{tabular}
  \centering
  \inputminted{js}{./code/example2_consult_account_response.jsch}
\end{frame}

\begin{frame}{Operation: deposit}
  \centering
  \begin{tabular}{| c | l |}
    \hline
    \textbf{Operation} & deposit \\ \hline
    \textbf{URI}       & http://localhost:5000/bank/users/\{owner\}/accounts/\{accountid\} \\ \hline
    \textbf{Method}    & POST     \\ \hline
    \textbf{Body}      & operation: ``deposit'', quantity: integer \\ \hline
  \end{tabular}
  \centering
  \inputminted{js}{./code/example2_deposit.jsch}
\end{frame}

\begin{frame}{Operation: deposit}
  \centering
  \begin{tabular}{| c | l |}
    \hline
    \textbf{Result}    & accountid: string, balance: integer, owner: string\\ \hline
    \textbf{Status}    & 201 \\ \hline
  \end{tabular}
  \centering
  \inputminted{js}{./code/example2_deposit_response.jsch}
\end{frame}

\begin{frame}{Operation: withdraw}
  \centering
  \begin{tabular}{| c | l |}
    \hline
    \textbf{Operation} & withdraw \\ \hline
    \textbf{URI}       & http://localhost:5000/bank/users/\{owner\}/accounts/\{accountid\} \\ \hline
    \textbf{Method}    & POST     \\ \hline
    \textbf{Body}      & operation: ``withdarw'', quantity: integer \\ \hline
  \end{tabular}
  \centering
  \inputminted{js}{./code/example2_withdraw.jsch}
\end{frame}

\begin{frame}{Operation: withdraw}
  \centering
  \begin{tabular}{| c | l |}
    \hline
    \textbf{Result}    & accountid: string, balance: integer, owner: string\\ \hline
    \textbf{Status}    & 201 \\ \hline
  \end{tabular}
  \centering
  \inputminted{js}{./code/example2_withdraw_response1.jsch}
\end{frame}

\begin{frame}
  \inputminted{js}{./code/example2_withdraw_response2.jsch}
\end{frame}

\begin{frame}{Dependencies}
  \begin{figure}
    \centering
    \includestandalone[mode=buildnew, scale=0.7]{./figures/figure-operations-dependency}
    \caption{Operation availability dependency}
    \label{fig:operation-dependence}
  \end{figure}
\end{frame}

\begin{frame}{Structure}
  \begin{figure}
    \centering
    \includegraphics[width=0.6\textwidth]{./figures/figure-example-2-account}
    \caption{File discovery structure}
    \label{fig:structure}
  \end{figure}

\end{frame}

\begin{frame}[standout]
  Demo
\end{frame}