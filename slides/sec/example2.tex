\section{Testing a web service dynamic state using jsongen}

\begin{frame}{Bank api operations}
  \begin{figure}
    \centering
    \includestandalone[mode=buildnew, width=\textwidth]{./figures/figure-api-6}%
  \end{figure}
\end{frame}

\begin{frame}{Objectives and API description}%
  \begin{itemize}
  \item The main objective of this example is to give a general idea
    of how to use the dynamic links descovering habilities of jsongen.

  \item In this example we will test the protocol of the whole bank API.%

  \item Let's revisit our possible operations:%
  \end{itemize}

  \centering
  \begin{tabular}{| l | l |}
    \hline
    \textbf{Operation} & \textbf{Resource identifiers} \\ \hline
    new user           & /bank/users/                  \\ \hline
    new account        & /bank/users/\{user\}/accounts/  \\ \hline
    consult account    & /bank/users/\{owner\}/accounts/\{accountid\}/ \\ \hline
    deposit            & /bank/users/\{owner\}/accounts/\{accountid\}/ \\ \hline
    withdraw           & /bank/users/\{owner\}/accounts/\{accountid\}/ \\ \hline
  \end{tabular}
  \centering

\end{frame}

\begin{frame}{Dynamic discovery of operations}%
  \begin{itemize}
  \item Jsongen can create sequences of operations with data received
    in previous requests.

  \item When jsogen validates a response, we can define a new link to
    explore within the JSON Schema.
  \item Our \texttt{create\_account} operation unlocks three
    operations over the account created:%
    \begin{itemize}
    \item balance
    \item deposit
    \item withdraw
    \end{itemize}
  \end{itemize}

\end{frame}

\begin{frame}{Operation: new account}

  We need a user in order to create a new account. This user is taken
  from the \texttt{new\_user} response:

  \centering
  \inputminted{js}{./code/example2_user_response-gen.jsch}
  % \vspace{5pt}
  % \hrule{}

  \pause
  % \centering
  % \begin{tabular}{| c | l |}
      %       \hline
      %       \textbf{Operation} & new account \\ \hline
      %       \textbf{URI}       & http://localhost:5000/bank/users/\{user\}/accounts/ \\ \hline
      %       \textbf{Method}    & POST     \\ \hline
      %       \textbf{Body}      & empty    \\ \hline
      %     \end{tabular}

  We create our next request with a reference to the \texttt{user}
  value returned:

  \centering
  \inputminted{js}{./code/example2_new_account.jsch}
\end{frame}

\begin{frame}{Operation: new account}
  \centering
  \begin{tabular}{| c | l |}
    \hline
    \textbf{Result}    & accountid: string, balance: integer, owner: string\\ \hline
    \textbf{Status}    & 201 \\ \hline
  \end{tabular}
  \centering
  \inputminted{js}{./code/example2_new_account_response.jsch}
\end{frame}


\begin{frame}{Operation: new account}
  Now we define the operations unlocked when we create an account.

  \centering
  \inputminted{js}{./code/example2_new_account_links.jsch}
\end{frame}

% \begin{frame}{Operation: get account balance}
%   \centering
%   \begin{tabular}{| c | l |}
      %       \hline
      %       \textbf{Operation} & consult account \\ \hline
      %       \textbf{URI}       & http://localhost:5000/bank/users/\{owner\}/accounts/\{accountid\} \\ \hline
      %       \textbf{Method}    & GET     \\ \hline
      %     \end{tabular}
      %       \centering
      %       \inputminted{js}{./code/example2_consult_account.jsch}
      %       \end{frame}

      %       \begin{frame}{Operation: get account balance}
      %       \centering
      %       \begin{tabular}{| c | l |}
      %       \hline
      %       \textbf{Result}    & accountid: string, balance: integer, owner: string\\ \hline
      %       \textbf{Status}    & 200 \\ \hline
      %     \end{tabular}
      %       \centering
      %       \inputminted{js}{./code/example2_consult_account_response.jsch}
      %       \end{frame}

      %       \begin{frame}{Operation: deposit}
      %       \centering
      %       \begin{tabular}{| c | l |}
      %       \hline
      %       \textbf{Operation} & deposit \\ \hline
      %       \textbf{URI}       & http://localhost:5000/bank/users/\{owner\}/accounts/\{accountid\} \\ \hline
      %       \textbf{Method}    & POST     \\ \hline
      %       \textbf{Body}      & operation: ``deposit'', quantity: integer \\ \hline
      %     \end{tabular}
      %       \centering
      %       \inputminted{js}{./code/example2_deposit.jsch}
      %       \end{frame}

      %       \begin{frame}{Operation: deposit}
      %       \centering
      %       \begin{tabular}{| c | l |}
      %       \hline
      %       \textbf{Result}    & accountid: string, balance: integer, owner: string\\ \hline
      %       \textbf{Status}    & 201 \\ \hline
      %     \end{tabular}
      %       \centering
      %       \inputminted{js}{./code/example2_deposit_response.jsch}
      %       \end{frame}

      %       \begin{frame}{Operation: withdraw}
      %       \centering
      %       \begin{tabular}{| c | l |}
      %       \hline
      %       \textbf{Operation} & withdraw \\ \hline
      %       \textbf{URI}       & http://localhost:5000/bank/users/\{owner\}/accounts/\{accountid\} \\ \hline
      %       \textbf{Method}    & POST     \\ \hline
      %       \textbf{Body}      & operation: ``withdarw'', quantity: integer \\ \hline
      %     \end{tabular}
      %       \centering
      %       \inputminted{js}{./code/example2_withdraw.jsch}
      %       \end{frame}

      %       \begin{frame}{Operation: withdraw}
      %       \centering
      %       \begin{tabular}{| c | l |}
      %       \hline
      %       \textbf{Result}    & accountid: string, balance: integer, owner: string\\ \hline
      %       \textbf{Status}    & 201 \\ \hline
      %     \end{tabular}
      %       \centering
      %       \inputminted{js}{./code/example2_withdraw_response1.jsch}
      %       \end{frame}

      %       \begin{frame}
      %       \inputminted{js}{./code/example2_withdraw_response2.jsch}
      %       \end{frame}

\begin{frame}{Operation availability dependency}
  \begin{figure}
    \centering
    \includestandalone[mode=buildnew, width=0.7\textwidth]{./figures/figure-operations-dependency}
  \end{figure}
\end{frame}

      %       \begin{frame}{Structure}
      %       \begin{figure}
      %       \centering
      %       \includestandalone[mode=buildnew, scale=0.7]{./figures/figure-example-2-account}
      %       \caption{File discovery structure}
      %       \label{fig:structure}
      %       \end{figure}

      %       \end{frame}

\begin{frame}{}{}
  \begin{center}
    \vspace{15pt}
    {\Huge Demo}
  \end{center}
\end{frame}